
\newcommand{\VAESolAa}{
%%%%%%%%%%%%%%%%%%%%%%%%%%%%%%%%%%%%
%%
%%.   YOUR SOLUTION FOR PROBLEM A BELOW THIS COMMENT
%%
%%%%%%%%%%%%%%%%%%%%%%%%%%%%%%%%%%%%
$$D_{KL}(p(x) \| q(x)) = - \int p(x) \bigg( \log \bigg\{ \frac{q(x)}{p(x)} \bigg\} \bigg)dx$$
}

\newcommand{\VAESolAb}{
%%%%%%%%%%%%%%%%%%%%%%%%%%%%%%%%%%%%
%%
%%.   YOUR SOLUTION FOR PROBLEM A BELOW THIS COMMENT
%%
%%%%%%%%%%%%%%%%%%%%%%%%%%%%%%%%%%%%
$$KL(p(x) \| q(x)) = - \int p(x) \bigg( \log \bigg\{ \frac{q(x)}{p(x)} \bigg\} \bigg)dx \geq - \log \int p(x) \frac{q(x)}{p(x)} = - \log \int q(x)dx = 0$$
}


\newcommand{\VAESolBa}{	
%%%%%%%%%%%%%%%%%%%%%%%%%%%%%%%%%%%%
%%
%%.   YOUR SOLUTION FOR PROBLEM B BELOW THIS COMMENT
%%
%%%%%%%%%%%%%%%%%%%%%%%%%%%%%%%%%%%%
\begin{align*}
\log p_\theta(x) &= \int_z q_\phi(z|x)\log\frac{p_\theta(x,z)}{q_\phi(z|x)}dz + \int_z q_\phi(z|x)\log\frac{q_\phi(z|x)}{p_\theta(z|x)}dz\\
&= \mathcal{L}(p_{\theta}, q_{\phi}) + D_{KL}(q_{\phi}, p_{\theta})\\
&\geq \mathcal{L}(p_{\theta}, q_{\phi})
\end{align*}

}

\newcommand{\VAESolBb}{	
%%%%%%%%%%%%%%%%%%%%%%%%%%%%%%%%%%%%
%%
%%.   YOUR SOLUTION FOR PROBLEM B BELOW THIS COMMENT
%%
%%%%%%%%%%%%%%%%%%%%%%%%%%%%%%%%%%%%
$\mathcal{L}(p_{\theta}, q_{\phi})$ is often referred to as empirical lower bound (ELBO).
}

\newcommand{\VAESolBc}{	
%%%%%%%%%%%%%%%%%%%%%%%%%%%%%%%%%%%%
%%
%%.   YOUR SOLUTION FOR PROBLEM B BELOW THIS COMMENT
%%
%%%%%%%%%%%%%%%%%%%%%%%%%%%%%%%%%%%%
It holds with equality if and only if $q_{\phi} = p_{\theta}$ for all x.
}


\newcommand{\VAESolC}{
%%%%%%%%%%%%%%%%%%%%%%%%%%%%%%%%%%%%
%%
%%.   YOUR SOLUTION FOR PROBLEM C BELOW THIS COMMENT
%%
%%%%%%%%%%%%%%%%%%%%%%%%%%%%%%%%%%%%
\begin{align*}
	D_{KL}(q(z|x) \|p(z)) &= \mathbb{E}[\log q(z|x) - p(z)]\\
	&= \frac{1}{2} \cdot (\sigma_z^2 + \mu_z^2 - 1 - \log (\sigma_z^2))
\end{align*}
}


\newcommand{\VAESolD}{
%%%%%%%%%%%%%%%%%%%%%%%%%%%%%%%%%%%%
%%
%%.   YOUR SOLUTION FOR PROBLEM D BELOW THIS COMMENT
%%
%%%%%%%%%%%%%%%%%%%%%%%%%%%%%%%%%%%%
\\It is more numerically stable to take exponent compared to computing log.
$$D_{KL}(q(z|x) \|p(z)) = \frac{1}{2} \cdot (\exp (\sigma_z^2) + \mu_z^2 - 1 - \sigma_z^2)$$

Furthermore, if we model just $\sigma_z^2$, it always outputs a non-negative number, but if we model $\sigma_z^2$ in log space, we can get all the number in $\mathbb{R}^1$, which will give us more options.
}

\newcommand{\VAESolE}{
%%%%%%%%%%%%%%%%%%%%%%%%%%%%%%%%%%%%
%%
%%.   YOUR SOLUTION FOR PROBLEM E BELOW THIS COMMENT
%%
%%%%%%%%%%%%%%%%%%%%%%%%%%%%%%%%%%%%
In a nutshell, reparametrization trick make sure we can backpropagate. We are given $z$ that is drawn from a distribution $q_\phi (z \vert x)$, and we want to take derivatives of a function of $z$ with respect to $\phi$, The reparametrization trick lets us backpropagate (take derivatives using the chain rule) with respect to $\phi$ through the objective (the ELBO) which is a function of samples of the latent variables $z$.
}


