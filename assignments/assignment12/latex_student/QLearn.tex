% !TEX root = exam.tex
\newcommand{\QLearnStudSolA}{
%%%%%%%%%%%%%%%%%%%%%%%%%%%%%%%%%%%%
%%
%%.   YOUR SOLUTION FOR PROBLEM A BELOW THIS COMMENT
%%
%%%%%%%%%%%%%%%%%%%%%%%%%%%%%%%%%%%%
$$Q^*(s, a) = \sum\limits_{s' \in S} P(s' | s, a) \bigg[ R(s,a,s') + \max \limits_{a' \in \mathcal{A}_{s'}}Q^*(s', a') \bigg]$$
\vspace{0cm}
}

\newcommand{\QLearnStudSolB}{
%%%%%%%%%%%%%%%%%%%%%%%%%%%%%%%%%%%%
%%
%%.   YOUR SOLUTION FOR PROBLEM A BELOW THIS COMMENT
%%
%%%%%%%%%%%%%%%%%%%%%%%%%%%%%%%%%%%%
\begin{itemize}
\item Obtain a sample mini-batch $\mathcal{B} \subseteq \mathcal{D}$, which $\mathcal{D} = \{(s,a,r,s') \}$
\item Compute target $$y_j = R(s,a,s') + \gamma \max\limits_{a' \in \mathcal{A}_s'} Q(s', a') \;\;\;\;\; \forall j \in \mathcal{B}$$
\item Use stochastic (semi-)gradient descent to optimize:
\begin{align*}
	Q(s, a) &= (1 - \alpha)Q(s, a) + \alpha y_j\\
    &= (1 - \alpha)Q(s, a) + \alpha (R(s,a,s') + \gamma \max\limits_{a' \in \mathcal{A}_s'} Q(s', a'))
\end{align*}
\end{itemize}

\vspace{0cm}
}

\newcommand{\QLearnStudSolC}{
%%%%%%%%%%%%%%%%%%%%%%%%%%%%%%%%%%%%
%%
%%.   YOUR SOLUTION FOR PROBLEM A BELOW THIS COMMENT
%%
%%%%%%%%%%%%%%%%%%%%%%%%%%%%%%%%%%%%
\\
Advantage of Epsilon greedy strategy is to explore (generate) more random samples rather than directly use the state space. (so it is more stochastic)
\vspace{0cm}
}

\newcommand{\QLearnStudSolD}{
%%%%%%%%%%%%%%%%%%%%%%%%%%%%%%%%%%%%
%%
%%.   YOUR SOLUTION FOR PROBLEM A BELOW THIS COMMENT
%%
%%%%%%%%%%%%%%%%%%%%%%%%%%%%%%%%%%%%
\begin{itemize}
\item Q-learning updates are incremental and do not converge quickly, so multiple passes with the same data is beneficial, especially when there is low variance in immediate outcomes $(r, s')$ given the same state, action pair $(s, a)$.
\item Better convergence behavior will be given when training function approximation. This is because the data is more like i.i.d. data assumed in most supervised learning convergence proofs.
\end{itemize}

\vspace{0cm}
}

\newcommand{\QLearnStudSolE}{
%%%%%%%%%%%%%%%%%%%%%%%%%%%%%%%%%%%%
%%
%%.   YOUR SOLUTION FOR PROBLEM A BELOW THIS COMMENT
%%
%%%%%%%%%%%%%%%%%%%%%%%%%%%%%%%%%%%%
\\
According to the solution in (b), the equation will become to:
\begin{align*}
	Q(s, a) &= (1 - \alpha)Q(s, a) + \alpha (R(s,a,s') + \gamma \max\limits_{a' \in \mathcal{A}_s'} Q(s', a'))\\
    &= 0.5 \times Q(s, a) + 0.5 \times (R(s,a,s') + 0.5  \times \max\limits_{a' \in \mathcal{A}_s'} Q(s', a'))
\end{align*}

\begin{itemize}
\item Step 1. $Q(S_1, a_1) = 0.5 \times 0 + 0.5 \times (-10 + 0.5 \times 0) = -5$
\item Step 2. $Q(S_1, a_2) = 0.5 \times 0 + 0.5 \times (-10 + 0.5 \times 0) = -5$
\item Step 3. $Q(S_2, a_1) = 0.5 \times 0 + 0.5 \times (18.5 + 0.5 \times -5) = 8$
\item Step 4. $Q(S_1, a_2) = 0.5 \times -5 + 0.5 \times (-10 + 0.5 \times 8) = -5.5$
\end{itemize}


\begin{center}       
\begin{tabular}[t]{|c c c|}
\hline
$Q$ & $S_{1}$ & $S_{2}$  \\ \hline
$a_{1}$&-5&.\\ \hline
$a_{2}$&.& .\\ \hline
\end{tabular}

\hfill

\begin{tabular}[t]{|c c c|}
\hline
$Q$ & $S_{1}$ & $S_{2}$  \\ \hline
$a_{1}$&-5&.\\ \hline
$a_{2}$&-5&.\\ \hline
\end{tabular}

\hfill

\begin{tabular}[t]{|c c c|}
\hline
$Q$ & $S_{1}$ & $S_{2}$  \\ \hline
$a_{1}$&-5&8\\ \hline
$a_{2}$&-5&. \\ \hline
\end{tabular}

\hfill

\begin{tabular}[t]{|c c c|}
\hline
$Q$ & $S_{1}$ & $S_{2}$  \\ \hline
$a_{1}$&-5&8\\ \hline
$a_{2}$&-5.5&. \\ \hline
\end{tabular}
\end{center}

\vspace{0cm}
}
 %The students have to fill this file to print the solution


% Problem Explanation:
% - first argument is the number of points
% - second argument is the title and the text
\examproblem{13}{Q-Learning
}

%%%%%%%%%%%%%%%%%%%%%%%%%%%%%%%%%%%%%%
%%%%%  BEGINNING OF SUBPROBLEMS LIST
\begin{enumerate}

% Subproblem description
\examproblempart{State the Bellman optimality principle as a function of the optimal Q-function $Q^{*}(s,a)$, the expected reward function $R(s,a,s')$ and the transition probability $P(s'|s,a)$, where $s$ is the current state, $s'$ is the next state and $a$ is the action taken in state $s$.\\
}
\bookletskip{0}   %in inches

% Solution box 
  \framebox[14.7cm][l]{
 \begin{minipage}[b]{14.2cm}
 \inbooklet{Your answer: \QLearnStudSolA}
  
 \solution{\QLearnSolA}
 \end{minipage}
 }
 
% Subproblem description 
 \examproblempart{In case the transition probability $P(s'|s,a)$ and the expected reward $R(s,a,s')$  are unknown, a stochastic  approach is used to approximate the optimal Q-function. After observing a transition of the form $(s,a,r,s')$, write down the update of the Q-function at the observed state-action pair $(s,a)$ as a function of the learning rate $\alpha$, the discount factor $\gamma$, $Q(s,a)$ and $Q(s',a')$.\\

}
\bookletskip{0}   %in inches

% Solution box 
  \framebox[14.7cm][l]{
 \begin{minipage}[b]{14.2cm}
 \inbooklet{Your answer: \QLearnStudSolB}
  
 \solution{\QLearnSolB}
 \end{minipage}
 }

% Subproblem description
\examproblempart{What is the advantage of an epsilon-greedy strategy? \\}
\bookletskip{0.0}   %in inches
 
% Solution box  
   \framebox[14.7cm][l]{
 \begin{minipage}[b]{14.2cm}
 \inbooklet{Your answer: \QLearnStudSolC}
  
 \solution{\QLearnSolC}
 \end{minipage}
 }

\newpage
% Subproblem description 
 \examproblempart{What is the advantage of using a replay-memory?  \\}
\bookletskip{0.0}   %in inches
 
% Solution box  
 \framebox[14.7cm][l]{
 \begin{minipage}[b]{14.2cm}
 \inbooklet{Your answer: \QLearnStudSolD}
  
 \solution{\QLearnSolD}
 \end{minipage}
 }
 

 
\bookletpage
% Subproblem description
 \examproblempart{Consider a system with two states $S_{1}$ and $S_{2}$ and two actions $a_{1}$ and $a_{2}$. You perform actions and observe the rewards and transitions listed below. Each step lists the current state, reward, action and resulting transition as: $S_{i};  R=r; a_{k}: S_{i} \rightarrow S_{j}  $. Perform Q-learning using a learning rate of $\alpha=0.5$ and a discount factor of $\gamma=0.5$ for each step by applying the formula from part (b). The Q-table entries are initialized to zero.  Fill in the tables below corresponding to the following four transitions. What is the optimal policy after having observed the four transitions?}
 
 \begin{enumerate}
\item $S_{1}$; $R=-10$; $a_{1}:S_{1}\rightarrow S_{1}$
\item  $S_{1}$; $R=-10$; $a_{2}:S_{1}\rightarrow S_{2}$
\item $S_{2}$; $R=18.5$; $a_{1}:S_{2}\rightarrow S_{1}$
\item $S_{1}$; $R=-10$; $a_{2}:S_{1}\rightarrow S_{2}$

\end{enumerate}

 

\bookletskip{0}   %in inches


\begin{table}[h]
\small       
\begin{tabular}[t]{|c c c|}
\hline
$Q$ & $S_{1}$ & $S_{2}$  \\ \hline
$a_{1}$&.&.\\ \hline
$a_{2}$&.& .\\ \hline
\end{tabular}
\hfill
\begin{tabular}[t]{|c c c|}
\hline
$Q$ & $S_{1}$ & $S_{2}$  \\ \hline
$a_{1}$&.&.\\ \hline
$a_{2}$&.&.\\ \hline
\end{tabular}
\hfill
\begin{tabular}[t]{|c c c|}
 \hline
 $Q$ & $S_{1}$ & $S_{2}$  \\ \hline
 $a_{1}$&.&.\\ \hline
 $a_{2}$&.&. \\ \hline
 \end{tabular}
 \hfill
 \begin{tabular}[t]{|c c c|}
 \hline
 $Q$ & $S_{1}$ & $S_{2}$  \\ \hline
 $a_{1}$&.&.\\ \hline
 $a_{2}$&.&. \\ \hline
\end{tabular}
\end{table}

% Solution box 
  \framebox[14.7cm][l]{
 \begin{minipage}[b]{14.2cm}
 \inbooklet{Your answer: \QLearnStudSolE}
  
 \solution{\QLearnSolE}
 \end{minipage}
 }
 
%%%%%%%%%%%% END OF SUBPROBLEMS LIST

\end{enumerate}