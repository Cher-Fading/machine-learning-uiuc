\newcommand{\GANStudSolA}{
	%%%%%%%%%%%%%%%%%%%%%%%%%%%%%%%%%%%%
	%%
	%%.   YOUR SOLUTION FOR PROBLEM A BELOW THIS COMMENT
	%%
	%%%%%%%%%%%%%%%%%%%%%%%%%%%%%%%%%%%%
	\\
	Original GAN minimizes a divergence/distance between probability distributions.
	
	$$\min_{G} \max_{D} V(D_{w}(x), G_{ \theta}(x)) = \mathbb{E}_{x \sim p_{data}}[\log D_{w}(x)] + \mathbb{E}_{x \sim p_{g}}[\log (1 - G_{ \theta}(x))]$$
	\vspace{0cm}
}

\newcommand{\GANStudSolB}{
	%%%%%%%%%%%%%%%%%%%%%%%%%%%%%%%%%%%%
	%%
	%%.   YOUR SOLUTION FOR PROBLEM A BELOW THIS COMMENT
	%%
	%%%%%%%%%%%%%%%%%%%%%%%%%%%%%%%%%%%%
	\begin{align}
		\max_{D} V(G, D) &= \mathbb{E}_{x \sim p_{data}}[\log D_{w}(x)] + \mathbb{E}_{x \sim p_{g}}[\log (1 - G_{ \theta}(x))]\\
		&= \mathbb{E}_{x \sim p_{data}} \bigg[ \log \frac{p_{data}(x)}{p_{data}(x) + p_{g}(x)} \bigg] + \mathbb{E}_{x \sim p_{g}} \bigg[ \log \frac{p_{g}(x)}{p_{data}(x) + p_{g}(x)} \bigg]
	\end{align}
	\vspace{0cm}
}

\newcommand{\GANStudSolC}{
	%%%%%%%%%%%%%%%%%%%%%%%%%%%%%%%%%%%%
	%%
	%%.   YOUR SOLUTION FOR PROBLEM A BELOW THIS COMMENT
	%%
	%%%%%%%%%%%%%%%%%%%%%%%%%%%%%%%%%%%%
	\\
	The optimal discriminator $D^*(x)$ is given by fixed $G$ (generator). Given any $G$, maximize $V(G, D)$
	\begin{align}
		V(G, D) &= \int_{x}p_{data}(x)\log (D(x)) dx + \int_{z}p_{z}(z)\log (1 - D(g(z))) dz\\
		&= \int_{x}p_{data}(x)\log (D(x)) + p_{g}(x) \log (1 - D(x))dx
	\end{align}
	The reason why transforming from Eq. (3) to Eq. (4) is since $x = g(z)$, we can replace $g(z)$ with variable $x$. Also in this case, $p_g$ is the distribution of $x$ (in generator).\\
	\\For any $(a, b) \in \mathbb{R}^2 \backslash \{0, 0\}$, the function $y \rightarrow a \log (y) + b \log (1 - y)$ achieves its maximum in $[0,1]$ at $ \frac{a}{a + b}$. So the optimal discriminator $D^*(x)$ will be given by:
	$$D^*(x) = \frac{p_{data}(x)}{p_{data}(x) + p_{g}(x)}$$
	\vspace{0cm}
}

\newcommand{\GANStudSolD}{
	%%%%%%%%%%%%%%%%%%%%%%%%%%%%%%%%%%%%
	%%
	%%.   YOUR SOLUTION FOR PROBLEM A BELOW THIS COMMENT
	%%
	%%%%%%%%%%%%%%%%%%%%%%%%%%%%%%%%%%%%
	\\
	For $p_G^* = p_{data}$, we can get $$D^*(x) = \frac{p_{data}(x)}{p_{data}(x) + p_{data}(x)} = \frac{1}{2},$$ so that $\max_{D} V(G, D) = \log \frac{1}{2} + \log \frac{1}{2} = -\log 4$\\
	\\
	From Eq. (2), we can get:
	\begin{align}
		\max_{D} V(G, D) &= \mathbb{E}_{x \sim p_{data}}[\log D^*(x)] + \mathbb{E}_{x \sim p_{g}}[\log (1 - D^*(x)]\\
		&= \mathbb{E}_{x \sim p_{data}} [- \log 2] + \mathbb{E}_{x \sim p_{g}} [- \log 2]\\
		&= - \log 4
	\end{align}
	
	Meanwhile, if we simply Eq. (2):
	\begin{align}
		C(G) &= \max_{D} V(G, D)\\
		&= \mathbb{E}_{x \sim p_{data}} \bigg[ \log \frac{p_{data}(x)}{p_{data}(x) + p_G^*(x)} \bigg] + \mathbb{E}_{x \sim p_G^*} \bigg[ \log \frac{p_G^*(x)}{p_{data}(x) + p_G^*(x)} \bigg]\\
		&= \int_x \Bigg( p_{data}(x) \log \Big( \frac{p_{data}(x)}{p_{data}(x) + p_G^*(x)} \Big)  + p_G^*(x) \log \Big( \frac{p_G^*(x)}{p_{data}(x) + p_G^*(x)} \Big)  \Bigg)dx\\
		&= \int_x \Bigg( p_{data}(x) \log \Big( \frac{2p_{data}(x)}{p_{data}(x) + p_G^*(x)} \Big)  + p_G^*(x) \log \Big( \frac{2p_G^*(x)}{p_{data}(x) + p_G^*(x)} \Big)  \Bigg)dx -\log 4\\
		&= - \log 4 + KL \bigg( p_{data} \Vert \frac{p_{data}(x) + p_G^*(x)}{2} \bigg) + KL \bigg( p_G^* \Vert \frac{p_{data}(x) + p_G^*(x)}{2} \bigg)\\
		&= - \log 4 + 2 \cdot JSD(p_{data}, p_G^*)
	\end{align}
	
	Since the Jensen-Shannon divergence between two distributions is always non-negative and equals to zero if and only if the two distributions are equal. So that if $C(G)$ wants to be the optimal value (global optimum), the only solution is:
	$$p_G^* = p_{data}$$
	\vspace{0cm}
}

\newcommand{\GANStudSolE}{
	%%%%%%%%%%%%%%%%%%%%%%%%%%%%%%%%%%%%
	%%
	%%.   YOUR SOLUTION FOR PROBLEM A BELOW THIS COMMENT
	%%
	%%%%%%%%%%%%%%%%%%%%%%%%%%%%%%%%%%%%
	\begin{align*}
		D_{KL}(\bP_1,\bP_2) &= \int \bP_1 \log \frac{\bP_1}{\bP_2}\\
		&= 0
	\end{align*}
	\begin{align*}
		D_{KL}(\bP_1,\bP_3) &= \int \bP_1 \log \frac{\bP_1}{\bP_3}\\
		&= 0
	\end{align*}
	\begin{align*}
		\bW_1(\bP_1,\bP_2) &= \sup \vert \mathbb{E}_{x \sim \bP_1}[f(x)] - \mathbb{E}_{y \sim \bP_2}[f(y)] \vert \\
		&= \sup \vert f(x) - f(y) \vert \\
		&= 1.5
	\end{align*}
	\begin{align*}
		\bW_1(\bP_1,\bP_3) &= \sup \vert \mathbb{E}_{x \sim \bP_1}[f(x)] - \mathbb{E}_{y \sim \bP_3}[f(y)] \vert \\
		&= \sup \vert f(x) - f(y) \vert \\
		&= 2
	\end{align*}
	
	where $f \in \mathcal{F}_L$, a natural class of smooth functions is the class of 1-Lipschitz functions, i.e.
	$$\mathcal{F}_L = \{f:f\texttt{continuous}, \vert f(x) - f(y) \vert  \leq \| x - y \| \}$$
	
	\vspace{0cm}
}
